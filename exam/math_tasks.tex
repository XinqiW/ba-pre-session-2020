\documentclass[10pt]{article}
 
\usepackage[margin=1in]{geometry} 
\usepackage{amsmath,amsthm,amssymb, graphicx, multicol, array, enumerate, gensymb}
\newcommand{\N}{\mathbb{N}}
\newcommand{\Z}{\mathbb{Z}}
 
\newenvironment{problem}[2][Problem]{\begin{trivlist}
\item[\hskip \labelsep {\bfseries #1}\hskip \labelsep {\bfseries #2.}]}{\end{trivlist}}

\begin{document}
 
\title{Mathematics problems}
\date{}
\maketitle

 \section{Elementary algebra}
 
\begin{problem}{1.1}
Simplify $$\frac{x^{32}}{x^9 \cdot x^2} \cdot \frac{x^7}{x^2}$$
\end{problem}

\begin{problem}{1.2}
Solve for $x$:
$$8^2 \cdot 4^x \cdot 2^x = 8^4$$
\end{problem}

\begin{problem}{1.3}
Calculate the missing value. If $\frac{x}{y}$ is 3, then $x^{-4}y^{4}=\dots$
\end{problem}

\begin{problem}{1.4}
Calculate
$$\frac{\sqrt{4^{15}}}{\sqrt{16^7}}$$
\end{problem}

\begin{problem}{1.5}
True or False ($x$ and $y$ and $z$ are real numbers):
\begin{enumerate}[(a)]
    \item $x+(y+z)=(y+x)+z$
    \item $y(x+z)=xy+zy$
    \item $x^{y+z}=x^z+x^y$
    \item $\frac{x^z}{x^y}=x^{y-z}$
\end{enumerate}
\end{problem}

\begin{problem}{1.6}
Find the solution set for the inequality below:
$$\ln(x) \ge e$$
\end{problem}

\section{Functions of one variable}

\begin{problem}{2.1 (Based on SYD 2.5.6)}
The relationship between temperatures measured in Celsius and Fahrenheit is linear. 0\degree C is equivalent to 32\degree F and 100\degree C is the same as 212\degree F.
 Which temperature is measured by the same number on both scales?
\end{problem}

\begin{problem}{2.2}
Take the following function $f(x)=3x-12$. Find y if $f(y)=0$.
\end{problem}

\begin{problem}{2.3}
Find all values of x that satisfy:
$$9^{x^2-6x+2}=81$$
\end{problem}

\begin{problem}{2.4}
Solve the following problem. If the annual GDP growth of a country is 3\%, how long does it take the economy to triple its GDP?
\end{problem}

\begin{problem}{2.5}
Calculate the following value
$$\log_{\pi}\left(\frac{1}{\pi^5} \right)$$
\end{problem}

\section{Calculus}

\begin{problem}{3.1}
Calculate the following sum
$$\sum\limits_{i=0}^{\infty} \left( \frac{1}{5^i}+0.3^i\right)$$
\end{problem}

\begin{problem}{3.2}
Find the following limit
$$\lim\limits_{x \rightarrow 5}\frac{x^2-25}{x-5}$$
\end{problem}

\begin{problem}{3.3}
Find the slope of the function $f(x)=x^3-4$ at $(-2,-12)$.
\end{problem}

\begin{problem}{3.4}
Find the derivative of the following function:
$$f(x)= \frac{x^5+3}{x^2-1}$$
\end{problem}

\begin{problem}{3.5}
Find the second derivative of the following function:
 $$f(x) = x^9+3$$
\end{problem}

\begin{problem}{3.6}
Is the function  $f(x)=\frac{1}{x}$ continuous at $0$? Why?
\end{problem}

\begin{problem}{3.7}
Consider the following function. Find all of its local minima, local maxima or inflection points. 
$$f(x)=4x^3-12x$$
\end{problem}

\begin{problem}{3.8}
Let $f(x,y)=x^3-y^2$. Calculate $f(2,3)$
\end{problem}

\begin{problem}{3.9}
Consider the following function: $f(x,y)=\ln(x-3y)$. For what combinations of $x$ and $y$ is this function defined?
\end{problem}

\begin{problem}{3.10}
Find the following partial derivative:
$$\frac{\partial}{\partial \, x} \left(x^5y^7+\frac{x^2}{y^3}\right)$$
\end{problem}

\begin{problem}{3.11}
Find the local maxima or minima of the following function:
$$f(x,y)=\sqrt{xy}-x-y$$
\end{problem}

\begin{problem}{3.12}
Solve the following constrained optimization problem using Lagrange's method:
$\max x^2y^2$ s.t. $2x+y=9$
\end{problem}

\section{Linear algebra}

\begin{problem}{4.1}
Take the following matrices:
$$A=\begin{bmatrix} 2 & 5\\ 2 & 1 \\ 7 & 6\end{bmatrix}$$
$$B=\begin{bmatrix} 1 & 0 & 1\\9 & 1 & 5\end{bmatrix}$$
What is $B \cdot A$?
\end{problem}

\begin{problem}{4.2}
Take the following matrices:
$$A=\begin{bmatrix} 5 & 3\\ 0 & 1 \\ 1 & 2\end{bmatrix}$$
$$B=\begin{bmatrix} 8 & 4 & 0\\2 & 1 & 2\end{bmatrix}$$
What is $A \cdot B$?
\end{problem}

\begin{problem}{4.3}
What is the transpose of the following matrix?
$$\begin{bmatrix}e & 93 & 4.7\\ 2 & 6.1 & 4.22 \\ 4 & \pi & 0\end{bmatrix}$$
\end{problem}

\begin{problem}{4.4}
Calculate the determinant of
$$\begin{bmatrix}2 & 6 \\ 2 & 8 \end{bmatrix} $$
\end{problem}

\section{Probability theory}

\begin{problem}{5.1}
You run an experiment where you toss a dice two times. Each time you get either 1, 2, 3, 4, 5 or 6. What is the sample space of your experiment?
\end{problem}

\begin{problem}{5.2}
Assume that in a certain country 0.1\% of the population uses a certain drug. You have a way to test drug use, which will give you a positive result in 98\% of the cases where the individual is indeed a drug user and a negative result in 99.7\% of the cases where the individual doesn't use the drug. What is the probability that someone with a positive drug test is indeed a drug user?
\end{problem}

\begin{problem}{5.3}
You run an experiment in which you toss a dice 20 times and record how many times you ended up with a 1, 2, 3, 4, 5 or 6. Your random variable is the number of times you ended up with a 5. What is  expected value of this random variable?
\end{problem}
\end{document}

